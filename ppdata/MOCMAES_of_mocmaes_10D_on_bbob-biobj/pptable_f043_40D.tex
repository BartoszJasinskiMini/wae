${\bf f_{43}}$ & \multicolumn{2}{@{}c@{}}{8.0 \quad} & \multicolumn{2}{@{}c@{}}{41747 \quad} & \multicolumn{2}{@{}c@{}}{$\infty$ \quad} & \multicolumn{2}{@{}c@{}}{$\infty$ \quad} & \multicolumn{2}{@{}c@{}}{$\infty$ \quad} & \multicolumn{2}{@{}c|@{}}{$\infty$} & 0 & /10\\
 & 71&(159) & \multicolumn{2}{@{}c@{}}{$\infty$} & \multicolumn{2}{@{}c@{}}{$\infty$} & \multicolumn{2}{@{}c@{}}{$\infty$} & \multicolumn{2}{@{}c@{}}{$\infty$} & \multicolumn{2}{@{}c|@{}}{$\infty$} & 0 & /10\\